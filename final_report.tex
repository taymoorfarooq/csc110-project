\documentclass[fontsize=11pt]{article}  
\usepackage{amsmath}  
\usepackage[utf8]{inputenc}  
\usepackage[margin=0.75in]{geometry}  
\usepackage{indentfirst}  
  
\title{CSC110 Written Report: A Data-Centric Analysis of the COVID-19 Pandemic's Impact on Different Sectors of the Canadian Economy}  
\author{Xi Chen, Taymoor Farooq, Se-Eum Kim, Henry Klinck}  
\date{Tuesday, December 14, 2021}  

\begin{document}

\maketitle

\section{Introduction}

The COVID-19 pandemic has undoubtedly had a significant negative impact on Canada's economy. Mandatory lock-downs, new business regulations and the Canadian population's overall social distancing behaviours all contributed to financial strain on many Canadian families and businesses.

Many businesses have suffered due to the lack of customers and many restrictions. Most in-person businesses such as restaurants, retail stores, and movie theatres have been closed. Large events such as sports games and concerts have been cancelled entirely. Travel and tourism have essentially been non-existent.

However, certain businesses have thrived from the pandemic. Most online and digital services including online shopping, streaming services, the video game industry and virtual meeting platforms have prospered thanks to most of the population staying indoors. For example, Netflix added 10 million new customers in the second quarter of 2020 (1). Pharmaceutical and medical companies have also done well.

One factor related to these kinds of economic trends is employment, and this is our motivation for choosing this topic. Unemployment has increased overall and as university students who will be looking for jobs in a few years, it is important for us to know if we are on the right track. 

So, our question is: \textbf{How has the pandemic affected different economic sectors in Canada?} To answer this question, we want to compare for each sector, the actual data gathered during the pandemic with the predicted values based on pre-pandemic data. Then we can see how certain trends have or have not changed due to the pandemic. 

After almost two years, the overall economy has mostly recovered, it has changed significantly. Some industries were able to adjust as remote working has become common. Other industries will probably not recover until the pandemic ends. This has major social consequences for the people who are working in certain industries.

Being able to distinguish between the two branches is necessary to deal with the pandemic. For individuals, this can influence decisions on employment or education. For society,  this is the first step to helping the people most affected negatively by the pandemic.

It is also valuable to determine how each economic sector was effected. The four economic sectors that this project will investigate are: the primary (production) sector, the secondary sector (encompasses companies that contribute to producing a finished, usable product or are involved in construction), the tertiary (service) sector, and the quaternary sector (associated with either the intellectual or knowledge-based economy). Based on the results of this project, we will be able to identify which economic sectors were the most impacted by the COVID-19 pandemic. 

\section{Datasets}

\section{Computational Overview }



\section{Instructions for Obtaining Data Sets and Running Program}

\section{Changes to Project Plan Since Proposal}

\section{Discussion Section}
*Note: 500-800 words

\section{References}

(1) Kafka, P. (2020, July 16). The pandemic has been great for Netflix. Vox. Retrieved December 12, 2021, from https://www.vox.com/recode/2020/7/16/21327451/netflix-covid-earnings-subscribers-q2. 


\end{document}

